\documentclass[12pt]{article}
\usepackage{amsmath}
\usepackage{amssymb}
\usepackage{amsthm}
\usepackage{enumerate}
\usepackage{hyperref}
\usepackage{xypic}
\usepackage{txfonts}
\usepackage{amsmath}
\usepackage{amssymb}
\usepackage{amscd}
\usepackage{amsmath, mathtools,amssymb}
\usepackage{amsfonts,semantic,colortbl,mathrsfs,stmaryrd}
\usepackage{enumerate}
\usepackage{multirow}
\usepackage{graphicx}
\date{Feb 14, 2012}
\newtheorem{thm}{Theorem}
\newtheorem{lemma}[thm]{Lemma}
\newtheorem{fact}[thm]{Fact}
\newtheorem{cor}[thm]{Corollary}
\newtheorem{eg}{Example}
\newtheorem{hw}{Problem}
\newcommand{\xor}{\otimes}
\newenvironment{sol}
  {\par\vspace{3mm}\noindent{\it Solution}.}
  {\qed}
\begin{document}
\begin{center}
{\LARGE\bf Homework 10}\\
\vspace{2mm}
\end{center}

\begin{hw}
Show that, for constant $p\in(0,1)$, almost no graph in $\mathcal{G}(n,p)$ has a separating complete subgraph.
\end{hw}
[Hint] \\
1. Recall the property $P_{i,j}$ from the slides\\
2.
You may need to recall the definitions:
\begin{itemize}
  \item \textbf{Separating subgraph}: Given $G=(V,E)$, and some $X\subseteq V\cup E$, we call $X$ a separating subgraph if there exists two vertices  $u,v \in V(G-X)$  such that $u,v$ are in the some component of $G$, while $u,v$ lie in two disconnected components of $G-X$ (i.e., $X$ separates $u$ and $v$).
  \item \textbf{Separating complete subgraph}: If the above subgraph $X$ is also a complete graph.
\end{itemize}

\begin{sol}
    Consider $P_{2,1}$. For any vertices $u,v,w$, there is a vertice $x$ that connects $u,v$ but disconnect $x$. Repeatedly, there is a vertex $y$ that connects $u,v$ but disconnects $x$.
    So, if there is a separating complete subgraph $X$, it connot hold $x,y$ concurrently. So, $u,v$ must be connected, which is contrasted. In a nutshell, almost no graph in $\mathcal{G}(n,p)$ has a separating complete subgraph.
\end{sol}

\begin{hw}
Consider $\mathbf{G}(n,p)$ with $p=\frac{1}{3n}$.
\begin{enumerate}[]
  \item Use the second moment method to show that with high probability there exists a simple path of length 10.
 % \item Argue that on the other hand, it is unlikely there exists any cycle of length 10.
\end{enumerate}
\end{hw}

\begin{sol}
    Suppose $x=\sum I_k, I_k$ is a random order to get $11$ vertices from $n$ vertices. The order of choosing vertices is the way to connecting a simple path. And 
    $$ I_k=\left\{
\begin{array}{rcl}
1       &      & I_k\text{ is a simple path of length 10}\\
0     &      & \text{others}
\end{array} \right. $$ 
    So, $E(x) = \binom{n}{11}\times 11! \times p^{10},\lim_{n \to \infty} E(x) = \frac{n(n-1)(n-2)\ldots (n-10)}{(3n)^{10}} = \infty $. And let $a = \binom{n}{11}\times 11!$.\\
    $E(x^2) = E(\sum I_k \sum I_j)= E(\sum I_k^2)+2E(\sum I_k I_j).$\\
    $E(\sum I_k^2) = E(\sum I_k) =E(x) = o(E^2(x))$.\\
    $E(\sum I_k I_j) = a(a-1)p^{20} = \theta (E^2(x))$.\\
    So, $E(x^2) = E(\sum I_k \sum I_j)= E(\sum I_k^2)+2E(\sum I_k I_j)=E^2(x)(1+o(1))$. So, $Var(x) = o(E^2(x))$. $x$ is almost surely greater than zero, which means there exists a simple path of length 10 with high probability.
\end{sol}


\begin{hw}
Prove that `the disappearance of isolated vertices in $\mathbf{G}(n,p)$' has a sharp threshold of $\frac{\ln n}{n}$.
\end{hw}
[Hint: John's book, theorem 8.6]

\begin{proof}
    Let $x$ be the number of isolated vertices in $G(n,p)$. Then, 
    \[ E(x) = n(1-p)^{n-1}. \]
    Let's consider $p = c \frac{\ln n}{n}$. Then,
    \[ \lim_{x \to \infty} E(x) = \lim_{x \to \infty} n(1-c\frac{\ln n}{n})^n = \lim_{x \to \infty} ne^{-c\ln n} =\lim_{x \to \infty} n^{1-c} .    \]
    So, when $c>1$, the expected number of isolated vertices goes to $0$, which means almost all graphs have no isolated vertices.
    When $c<1$, we will consider the second moment method.\\
    Let $x = I_1+I_2+\ldots +I_n$ where $I_i$ is a binary variable which means whether vertex $i$ is an isolated vertex. Then, $E(x^2) = \sum_{i = 1}^{n} E(I_i^2) + 2\sum_{i<j} E(I_iI_j)  $.\\
    $\sum_{i = 1}^{n} E(I_i^2) = \sum_{i = 1}^{n} E(I_i) = E(x)$.\\
    $2\sum_{i<j} E(I_iI_j)=n(n-1)E(I_iI_j)= n(n-1)(1-p)^{2(n-1)-1}$.\\
    So, $E(x^2) = E(x) + n(n-1)(1-p)^{2(n-1)-1}$. For $p = c\frac{\ln n}{n}$ and $c<1$,
    \[ \lim_{x \to \infty} \frac{E(x^2)}{E^2(x)} = \lim_{x \to \infty} \frac{n(1-p)^{n-1}+n(n-1)(1-p)^{2(n-1)-1}}{n^2(1-p)^{2(n-1)}} = \lim_{x \to \infty} [\frac{1}{n^{1-c}}+(1-\frac{1}{n})\frac{1}{1-c\frac{\ln n}{n}}]   \]
    \[= \lim_{x \to \infty} (1+c\frac{\ln n}{n}) = o(1)+1 .\]
    So, $Var(x)= E(x^2) - E^2(x) = o(E^2(x))$, which means $x$ is almost surely larger than $0$ when $c<1$. In a nutshell, this statement has to be proved.
\end{proof}


\begin{hw}(Optional)

\begin{enumerate}

\item Prove that the threshold for the existence of cycles in $\mathcal{G}(n,p)$ is $p=\frac{1}{n}$.
\item Search the \emph{World Wide Web} to find some real world graphs in machine readable form or data bases that could automatically be converted to graphs.

\begin{enumerate}
  \item Plot the degree distribution of each graph.
  \item Compute the average degree of each graph.
  \item Count the number of connected components of each size in each graph.
  \item Describe what you find.
\end{enumerate}
\item Create a simulation (an animation) to  show the evolution of the $\mathcal{G}(n,p)$ (Erd\"{o}s-R\'{e}nyi) random graph as its density $p$ is gradually increased. Observe the phase transitions for trees of increasing orders, followed by the emergence of the giant component, etc.
\end{enumerate}
\end{hw}



\end{document}

%%% Local Variables:
%%% mode: tex-pdf
%%% TeX-master: t
%%% End: