\documentclass[12pt]{article}
\usepackage{amsmath}
\usepackage{amssymb}
\usepackage{amsthm}
\usepackage{enumerate}
\usepackage{hyperref}
\usepackage{xypic}
\usepackage{txfonts}
\usepackage{amsmath}
\usepackage{amssymb}
\usepackage{amscd}
\usepackage{amsmath, mathtools,amssymb}
\usepackage{amsfonts,semantic,colortbl,mathrsfs,stmaryrd}
\usepackage{enumerate}
\usepackage{multirow}
\usepackage{graphicx}
\date{Feb 14, 2012}


\newtheorem*{solution}{Solution}
\newtheorem{thm}{Theorem}
\newtheorem{lemma}[thm]{Lemma}
\newtheorem{fact}[thm]{Fact}
\newtheorem{cor}[thm]{Corollary}
\newtheorem{eg}{Example}
\newtheorem{hw}{Problem}
\newcommand{\xor}{\otimes}
\newenvironment{sol}
  {\par\vspace{3mm}\noindent{\it Solution}.}
  {\qed}
\begin{document}
\begin{center}
{\LARGE\bf Homework 3}\\
\vspace{2mm}
\footnotesize{$*$ Name:\underline{Xin Xu}  \quad Student ID:\underline{519021910726} \quad Email: \underline{xuxin20010203@sjtu.edu.cn}}
\vspace{2mm}

\end{center}
\begin{hw}Prove the formula
\begin{enumerate}
\item
  ${r\choose r} +{r+1 \choose r}+{r+2 \choose r}+\cdots+{n\choose r} ={n+1 \choose r+1}$

 \item $\sum_{k=0}^{n} {m+k-1 \choose k} ={n+m \choose n}$

\end{enumerate}
\end{hw}

\begin{proof}
    \begin{enumerate}
        \item ${r\choose r} +{r+1 \choose r}+{r+2 \choose r}+\cdots+{n\choose r}$
        \\$={r+1\choose r+1} +{r+1 \choose r}+{r+2 \choose r}+\cdots+{n\choose r}$
        \\$={r+2\choose r+1} +{r+2 \choose r}+\cdots+{n\choose r}$
        \\$={n+1\choose r+1}$
        \item $\sum_{k=0}^{n} {m+k-1 \choose k}$
        \\$={m-1\choose 0}+{m\choose 1}+{m+1\choose 2}+\cdots+{m+n-1\choose n}$
        \\$={m\choose 0}+{m\choose 1}+{m+1\choose 2}+\cdots+{m+n-1\choose n}$
        \\$={m+1\choose 1}+{m+1\choose 2}+\cdots+{m+n-1\choose n}$
        \\$={m+n\choose n}$
    \end{enumerate}
\end{proof}
\begin{hw}
For natural numbers $m\leq n$ calculate (i.e. express by a simple formula not containing a sum) $\sum_{k=m}^n {k \choose m}{n \choose k}$.
\end{hw}

\begin{solution}
    By the meaning of combinatorial counting, this formula illustrates a situation where we pick $k$ people to form a group from the total $n$ people, and appoint $m$ people in the group as leaders.
    Then, the possible number of this kind of appointments is the answer. We should note that different groups and different numbers mean different appointments. So, $\sum_{k=m}^n {k \choose m}{n \choose k}$ can be translated into:
    ${n\choose m}\sum_{k=0}^n {n-m\choose k}$. And ${n\choose m}$ means the possible appointments of the leaders, $\sum_{k=0}^n {n-m\choose k}$ means the possible situations for other group members. So,
    \\$\sum_{k=m}^n {k \choose m}{n \choose k}$
    \\$={n\choose m}\sum_{k=0}^n {n-m\choose k}$
    \\$={n\choose m}(1+1)^{n-m}$
    \\$={n\choose m}2^{n-m}$
\end{solution}

\begin{hw}
Calculate (i.e. express by a simple formula not containing a sum)
\begin{enumerate}
  \item $\sum_{k=1}^n {k\choose m}\frac{1}{k}$
  \item $\sum_{k=0}^n{k\choose m}k$
\end{enumerate}
\end{hw}

\begin{solution}
    \begin{enumerate}
        \item $\sum_{k=1}^n {k\choose m}\frac{1}{k}$
        \\$=\frac{k!}{(k-m)!m!}\times \frac{1}{k} $
        \\$=\frac{(k-1)!}{(k-m)!(m-1)!}\times \frac{1}{m} $
        \\$=\frac{1}{m}\sum_{k=1}^n {k-1\choose m-1}$
        \\$=\frac{1}{m}{n\choose m}$
        \item $\sum_{k=0}^n{k\choose m}k$
        \\$=\sum_{k=0}^n{k\choose m}(k+1)-\sum_{k=0}^n{k\choose m}$
        \\$=\frac{(k+1)!}{(k-m)!m!}-{n+1\choose m+1}$
        \\$=(m+1)\frac{(k+1)!}{(k-m)!(m+1)!}-{n+1\choose m+1}$
        \\$=(m+1)\sum_{k=0}^n {k+1\choose m+1}-{n+1\choose m+1}$
        \\$=(m+1){n+2\choose m+2}-{n+1\choose m+1}$
    \end{enumerate}
\end{solution}

\begin{hw}
\begin{enumerate}[(a)]
\item Using \emph{\textbf{Problem 1.}} for $r=2$, calculate the sum $\sum_{i=2}^n i(i-1)$ and $\sum_{i=1}^n i^2$.
\item Use $(a)$ and \emph{\textbf{Problem 1.}}  for $r=3$, calculate $\sum_{i=1}^n i^3$.
\end{enumerate}
\end{hw}

\begin{solution}
    \begin{enumerate}
        \item $\sum_{i=2}^n i(i-1)$
        \\$=2\times \frac{\sum_{i=2}^n i(i-1)}{2\times 1}$
        \\$=2\times ({r\choose r} +{r+1 \choose r}+{r+2 \choose r}+\cdots+{n\choose r})$, $r=2$
        \\$=2\times {n+1\choose r+1}$, using \textbf{Problem 1.}. And $r=2$.
        \\$=2{n+1\choose 3}$
        \\$\sum_{i=1}^n i^2$
        \\$=\sum_{i=2}^n i(i-1)+1^2+\sum_{i=2}^n i$
        \\$=2{n+1\choose 3}+1+\frac{(n+2)(n-1)}{2}$
        \\$=2{n+1\choose 3}+\frac{n(n+1)}{2}$
        \item $\sum_{i=1}^n i^3$
        \\$=\sum_{i=1}^n i(i-1)(i-2)+3i^2-2i$
        \\$=6\times \frac{\sum_{i=1}^n i(i-1)(i-2)}{3\times 2\times 1}+6{n+1\choose 3}+\frac{n(n+1)}{2}$
        \\$=6{n+1\choose 4}+6{n+1\choose 3}+\frac{n(n+1)}{2}$
        \\$=6{n+2\choose 4}+\frac{n(n+1)}{2}$
    \end{enumerate}
\end{solution}

\begin{hw}
How many functions $f:\{1,2,\ldots, n\}\rightarrow \{1,2,\ldots, n\}$ are there that are \emph{monotone}; that is, for $i<j$ we have $f(i)\leq f(j)$?
\end{hw}

\begin{solution}
    We can transfer this question into another question: for any integer $1\leqslant i\leqslant n, x_i\geqslant 0, x_1+x_2+\cdots +x_n=n$. And $x_i$ means the number of $x$ that satisfies $f(x)=i$. So, we can easily know that the answer is ${2n-1\choose n-1}$.
\end{solution}

\end{document}

%%% Local Variables:
%%% mode: tex-pdf
%%% TeX-master: t
%%% End: